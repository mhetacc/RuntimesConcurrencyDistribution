\section{Introduction}

From sharing spreadsheets between a handful of laptops in a small basement office, through large-scale rendering on a supercomputer, to the entire global finance system, distributed computing has become an essential component of the modern world that we almost take for granted: nowadays, what most people need a computer for can be done in the browser thanks to services like email clients, cloud calendars, media streaming platforms and web-based office suites (like Google Docs) that expose word editors, spreadsheets managers, presentations programs and more, all while being constantly synchronized to the cloud, which not only ensures data persistence and availability, but also enables sharing and collaboration between users. 

It does not end here: other examples of distributed applications include cloud storage services like Dropbox, Google Drive or OneDrive, streaming services like Netflix, YouTube or Spotify, distributed computing like blockchain technologies or AWS, online banking services (the banking system itself is distributed since way before), social networks, and even maritime and aircraft traffic control systems.
Moreover, the rise of the gaming industry played a significant role in pushing forward distribution: in 2024 the gaming market revenue was estimated to be $187.7$ billion U.S. dollars \cite{newzoo}, making it a hefty slice of the pie that is the entertainment industry \cite{pwc}, with $111$ billions generated by free-to-play games \cite{f2prevenue} ($70$ billions from social and casual games alone\cite{casualgames}), which interests us since their business model often relies on cosmetics, game passes and advertisements, forcing them to be constantly on-line. 


Let's now define what distribution \textit{is}: a distributed system is a computer system whose inter-communicating components are located on different networked computers \cite{tanenbaum2017distributed} \cite{Apt2009}, which coordinate their actions via message-passing to achieve a common goal. There are three significant challenges to overcome: maintaining components' concurrency \footnote{Concurrency refers to the ability of a system to execute multiple tasks through simultaneous execution or time-sharing (context switching), sharing resources and managing interactions. It improves responsiveness, throughput, and scalability \cite{OSconcepts} \cite{computerOrganization} \cite{george_coulouris_distributed_2012} \cite{parallelComputing} \cite{parallelDistributedHandbook}.}, eliminating global-lock reliance and managing the independent failure of components, all while ensuring scalability (often the purpose is scaling itself) and transparency to the user, meaning interactions with any exposed interface must be done while being unaware of the complexity behind them.

Most importantly, shared consensus must be guaranteed: it does not require much thought to see that all servers in a cluster should agree on one or more shared values, lest becoming a collection of un-related components that have little to do with collaboration (thus distribution). In the most traditional single-value consensus protocols, such as Paxos \cite{paxos}, cooperating nodes agree on a single value (e.g., an integer), while multi-value alternatives like Raft \cite{raft} aim to agree on a series of values (i.e., a log) growing over time forming a sort-of cluster's history. It is worth noting that both goals are hindered by the intrinsically asynchronous nature of real-world communication, which make it impossible to achieve consensus via deterministc algorithms, as stated by Fischer, Lynch and Paterson in their FLP impossibility theorem \cite{flp}. Thankfully this can be circumvented by injecting some degree of randomness.

The concepts and examples we mentioned so far allow us to finally present the goal of this project: we will create a simplified clone of Travian \footnote{Travian: Legends is a persistent, browser-based, massively multiplayer, online real-time strategy game developed by the German software company Travian Games. It was originally written and released in June 2004 as "Travian" by Gerhard Müller. Set in classical antiquity, Travian: Legends is a predominantly militaristic real-time strategy game. Source: \url{https://www.travian.com/international}}, an old real-time \footnote{Real-time games progresses in a continuous time frame, allowing all players (human or computer-controlled) to play at the same time. By contrast, in turn-based games players wait for their turn to play.} player-versus-player \footnote{Player-versus-player (PvP) is a type of game where real human players compete against each other, opposed to player-versus-environment (PvE) games, where players face computer-controlled opponents.} strategy game \footnote{Strategy video game is a major video game genre that focuses on analyzing and strategizing over direct quick reaction in order to secure success. Although many types of video games can contain strategic elements, the strategy genre is most commonly defined by a primary focus on high-level strategy, logistics and resource management. \cite{rollings2003andrew}}, where players build their own city and wage war on one another (less wrinkly readers may be more familiar with the modern counterpart Clash of Clans \footnote{Clash of Clans: \url{https://supercell.com/en/games/clashofclans/}}), built with Pygame \footnote{Pygame: \url{https://www.pygame.org/docs/}}, a Python library that creates and manages all necessary components to run a game such as game-engine, graphical user interface, sounds, plyer inputs and the like, where each player resides in a separate server (or node) that communicate with the others via an algorithm modelled after Raft's specifications. 

This choice follows the author's interest in exploring Raft capabilities and ease of implementation in a fun and novel way, using a language that while extremely popular is seldom used in such a fashion.\\
Both game and algorithm implementations have been reduced to a reasonably complex proof of concept to keep the project scope manageable: it is possible to instantiate games up to five players, each of which is restricted to the only action of attacking the others, while Raft's functionalities are limited to log replication and overwriting. \\
Experiments were conducted to evaluate both game responsiveness and the communication algorithm correctness. \\
All source code is visible at the following link: \url{https://github.com/mhetacc/RuntimesConcurrencyDistribution/blob/main/raftian/raftian.py}.