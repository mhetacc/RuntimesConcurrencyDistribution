%%
%% This is file `sample-manuscript.tex',
%% generated with the docstrip utility.
%%
%% The original source files were:
%%
%% samples.dtx  (with options: `all,proceedings,bibtex,manuscript')
%% 
%% IMPORTANT NOTICE:
%% 
%% For the copyright see the source file.
%% 
%% Any modified versions of this file must be renamed
%% with new filenames distinct from sample-manuscript.tex.
%% 
%% For distribution of the original source see the terms
%% for copying and modification in the file samples.dtx.
%% 
%% This generated file may be distributed as long as the
%% original source files, as listed above, are part of the
%% same distribution. (The sources need not necessarily be
%% in the same archive or directory.)
%%
%%
%% Commands for TeXCount
%TC:macro \cite [option:text,text]
%TC:macro \citep [option:text,text]
%TC:macro \citet [option:text,text]
%TC:envir table 0 1
%TC:envir table* 0 1
%TC:envir tabular [ignore] word
%TC:envir displaymath 0 word
%TC:envir math 0 word
%TC:envir comment 0 0
%%
%% The first command in your LaTeX source must be the \documentclass
%% command.
%%
%% For submission and review of your manuscript please change the
%% command to \documentclass[manuscript, screen, review]{acmart}.
%%
%% When submitting camera ready or to TAPS, please change the command
%% to \documentclass[sigconf]{acmart} or whichever template is required
%% for your publication.
%%
%%
\documentclass[manuscript,screen,review]{acmart}
%%
%% \BibTeX command to typeset BibTeX logo in the docs
\AtBeginDocument{%
  \providecommand\BibTeX{{%
    Bib\TeX}}}

%% Rights management information.  This information is sent to you
%% when you complete the rights form.  These commands have SAMPLE
%% values in them; it is your responsibility as an author to replace
%% the commands and values with those provided to you when you
%% complete the rights form.
\setcopyright{acmlicensed}
\copyrightyear{2018}
\acmYear{2018}
\acmDOI{XXXXXXX.XXXXXXX}
%% These commands are for a PROCEEDINGS abstract or paper.
%\acmConference[Conference acronym 'XX]{Make sure to enter the correct
%  conference title from your rights confirmation email}{June 03--05,
%  2018}{Woodstock, NY}
%%
%%  Uncomment \acmBooktitle if the title of the proceedings is different
%%  from ``Proceedings of ...''!
%%
\acmBooktitle{Raftian: Pygame Meets Raft,
  TODO 03--05, 2025, Padua, IT}
\acmISBN{978-1-4503-XXXX-X/2018/06}


%%
%% Submission ID.
%% Use this when submitting an article to a sponsored event. You'll
%% receive a unique submission ID from the organizers
%% of the event, and this ID should be used as the parameter to this command.
%%\acmSubmissionID{123-A56-BU3}

%%
%% For managing citations, it is recommended to use bibliography
%% files in BibTeX format.
%%
%% You can then either use BibTeX with the ACM-Reference-Format style,
%% or BibLaTeX with the acmnumeric or acmauthoryear sytles, that include
%% support for advanced citation of software artefact from the
%% biblatex-software package, also separately available on CTAN.
%%
%% Look at the sample-*-biblatex.tex files for templates showcasing
%% the biblatex styles.
%%

%%
%% The majority of ACM publications use numbered citations and
%% references.  The command \citestyle{authoryear} switches to the
%% "author year" style.
%%
%% If you are preparing content for an event
%% sponsored by ACM SIGGRAPH, you must use the "author year" style of
%% citations and references.
%% Uncommenting
%% the next command will enable that style.
%%\citestyle{acmauthoryear}

% Default fixed font does not support bold face
\DeclareFixedFont{\ttb}{T1}{txtt}{bx}{n}{7} % for bold
\DeclareFixedFont{\ttm}{T1}{txtt}{m}{n}{7}  % for normal

% Custom colors
\usepackage{color}
\definecolor{deepblue}{rgb}{0,0,0.5}
\definecolor{deepred}{rgb}{0.6,0,0}
\definecolor{deepgreen}{rgb}{0,0.5,0}
\definecolor{lightpurple}{rgb}{0.6,0.4,0.8}  % New color for comments

\usepackage{listings}

% Python style for highlighting
\newcommand\pythonstyle{\lstset{
language=Python,
basicstyle=\ttm,
morekeywords={self},              % Add keywords here
keywordstyle=\ttb\color{deepblue},
emph={MyClass,__init__},          % Custom highlighting
emphstyle=\ttb\color{deepred},    % Custom highlighting style
stringstyle=\color{deepgreen},
commentstyle=\ttm\color{lightpurple},  % <- This sets comment font and color
frame=tb,                         % Any extra options here
showstringspaces=false,
}}


% Python environment
\lstnewenvironment{python}[1][]
{
\pythonstyle
\lstset{#1}
}
{}

% Python for external files
\newcommand\pythonexternal[2][]{{
\pythonstyle
\lstinputlisting[#1]{#2}}}

% Python for inline
\newcommand\pythoninline[1]{{\pythonstyle\lstinline!#1!}}

\usepackage{fontawesome5}

%%
%% end of the preamble, start of the body of the document source.
\begin{document}

%%
%% The "title" command has an optional parameter,
%% allowing the author to define a "short title" to be used in page headers.
\title{Raftian: Pygame Meets Raft}

%%
%% The "author" command and its associated commands are used to define
%% the authors and their affiliations.
%% Of note is the shared affiliation of the first two authors, and the
%% "authornote" and "authornotemark" commands
%% used to denote shared contribution to the research.
\author{Tullio Vardanega}
\authornote{Both authors contributed equally to this research.}
\email{tullio.vardanega@unipd.it}
\orcid{0000-0002-0089-0889}
\author{Marco Bellò}
\authornotemark[1]
\email{marco.bello.3@studenti.unipd.it}
\affiliation{%
  \institution{University of Padua}
  \city{Padua}
  \country{Italy}
}

%%
%% By default, the full list of authors will be used in the page
%% headers. Often, this list is too long, and will overlap
%% other information printed in the page headers. This command allows
%% the author to define a more concise list
%% of authors' names for this purpose.
\renewcommand{\shortauthors}{Vardanega and Bellò}

%%
%% The abstract is a short summary of the work to be presented in the
%% article.
\begin{abstract}
  The problem of shared consensus in a distributed system is both older than a millennium and more relevant than ever: while in the Aegean island of Paxos the challenge was to keep track of the many laws passed in a parliament where legislators had better to do than attend sessions full-time, nowadays the ubiquity of web-based architectures and applications (from a simple cloud storage to the whole banking system) gives daily headaches to developers and system administrators alike.
  To reduce ibuprofen consumption in the IT sector, Ongaro and Ousterhout devised an easily understandable and implementable alternative to the Paxos algorithm, namely the Raft consensus algorithm, which we employ in this work to implement communication via XML-RPC between multiple Pygame applications, each running a game instance that aims to simulate a simplified version of a real-time strategy game. 
\end{abstract}


%%
%% The code below is generated by the tool at http://dl.acm.org/ccs.cfm.
%% Please copy and paste the code instead of the example below.
%%
\begin{CCSXML}
<ccs2012>
   <concept>
       <concept_id>10010520.10010575.10011743</concept_id>
       <concept_desc>Computer systems organization~Fault-tolerant network topologies</concept_desc>
       <concept_significance>500</concept_significance>
       </concept>
   <concept>
       <concept_id>10010147.10011777.10011778</concept_id>
       <concept_desc>Computing methodologies~Concurrent algorithms</concept_desc>
       <concept_significance>500</concept_significance>
       </concept>
   <concept>
       <concept_id>10010147.10010919.10010172.10003824</concept_id>
       <concept_desc>Computing methodologies~Self-organization</concept_desc>
       <concept_significance>500</concept_significance>
       </concept>
   <concept>
       <concept_id>10010147.10010371.10010387.10010391</concept_id>
       <concept_desc>Computing methodologies~Graphics input devices</concept_desc>
       <concept_significance>300</concept_significance>
       </concept>
 </ccs2012>
\end{CCSXML}

\ccsdesc[500]{Computer systems organization~Fault-tolerant network topologies}
\ccsdesc[500]{Computing methodologies~Concurrent algorithms}
\ccsdesc[500]{Computing methodologies~Self-organization}
\ccsdesc[300]{Computing methodologies~Graphics input devices}

%%
%% Keywords. The author(s) should pick words that accurately describe
%% the work being presented. Separate the keywords with commas.
\keywords{Python, Raft, Distribution, Consensus, Gaming, Multiplayer, Multithreading, RPC, Pygame}

\received{20 February 2007}
\received[revised]{12 March 2009}
\received[accepted]{5 June 2009}

%%
%% This command processes the author and affiliation and title
%% information and builds the first part of the formatted document.
\maketitle

\section{Introduction}

% What and Why distribution
%Paxos
%Raft
%Online gaming
%Raftian
% Scope and Purpose of the investigation 


\section{Python} \label{sec:python}

We will now discuss the technologies employed in the development the the project (section \ref{sec:python}) as well as presenting our implementation of the algorithm Raft (section \ref{sec:raft}).

Python is a high-level, dynamically typed and interpreted programming language that is often used for scripting, data analysis and developing small applications, making it a non-obvious choice for this project, which is nothing of the above. 

As a language, it has two main of advantages compared to others: first of all it is undoubtedly the most popular and widely used in the world \cite{tiobe} \cite{ieeeSpect} which translates to abundant documentation and resources, and secondly it has a huge ecosystem of libraries that implement all the functionalities we need for this project, namely: \textit{xmlrpc} for the remote procedure calls (RPCs), \textit{threading} to handle local concurrency and \textit{Pygame} to manage everything game-related.

\subsection{Remote Procedure Calls} \label{sec:xmlrpc}

In Rafts' specifications it is stated that nodes communicate with each other via remote procedure calls \cite{raft}, which in distributed computing is when a program causes a procedure (or subroutine) to execute in another address space (commonly on another computer on a shared network) calling it as if it were local (that is, the programmer writes the same code whether the subroutine is local or remote).

There are many libraries that implement this functionality, like gRPC (\url{https://grpc.io/}) which is a high performance open source RPC framework used by many big players, like Netflix \footnote{Netflix Ribbon is an Inter Process Communication library built in software load balancers: \url{https://github.com/Netflix/ribbon}} and Cockroach Labs \footnote{Cockroach Labs is the company behind CockroachDB, a highly resilient distributed database: \url{https://www.cockroachlabs.com/}}, available for many languages (Python included), but we opted for the standard library \textit{xmlrpc} \footnote{XML-RPC is a Remote Procedure Call method that uses XML passed via HTTP as a transport: \url{https://docs.python.org/3/library/xmlrpc.html}} thanks to its promised simplicity and ease of use. 

The library provides both server and client implementations, encapsulating the former in its own loop, while the latter can be fired as needed allowing a bit more flexibility in its usage.

In the following code \verb|client| encapsulates a server living in its own networked location and is used to call the remote procedure \verb|test_foo(int)| like it was a local one.

\begin{python}
with xmlrpc.client.ServerProxy('http://localhost:8000', allow_none=True) as client:
    print(client.test_foo(42)) # print returned value
\end{python}

The server must be instantiated and left alive, and all remote procedure calls registered with \verb|register_function|.

\begin{python}
with SimpleXMLRPCServer (('localhost', 8000)) as server:
    def test_foo(number):
        return f'The number is {number}'

    server.register_function(test_foo)  
    server.serve_forever() # keep server alive
\end{python}

For this project we extend \textit{SimpleXMLRPCServer} to create a class that implements the Raft protocol (more details in section \ref{sec:implementation})

\section{Acknowledgments}

Identification of funding sources and other support, and thanks to
individuals and groups that assisted in the research and the
preparation of the work should be included in an acknowledgment
section, which is placed just before the reference section in your
document.

This section has a special environment:
\begin{verbatim}
  \begin{acks}
  ...
  \end{acks}
\end{verbatim}
so that the information contained therein can be more easily collected
during the article metadata extraction phase, and to ensure
consistency in the spelling of the section heading.

Authors should not prepare this section as a numbered or unnumbered {\verb|\section|}; please use the ``{\verb|acks|}'' environment.

\section{Appendices}

If your work needs an appendix, add it before the
``\verb|\end{document}|'' command at the conclusion of your source
document.

Start the appendix with the ``\verb|appendix|'' command:
\begin{verbatim}
  \appendix
\end{verbatim}
and note that in the appendix, sections are lettered, not
numbered. This document has two appendices, demonstrating the section
and subsection identification method.

\section{Multi-language papers}

Papers may be written in languages other than English or include
titles, subtitles, keywords and abstracts in different languages (as a
rule, a paper in a language other than English should include an
English title and an English abstract).  Use \verb|language=...| for
every language used in the paper.  The last language indicated is the
main language of the paper.  For example, a French paper with
additional titles and abstracts in English and German may start with
the following command
\begin{verbatim}
\documentclass[sigconf, language=english, language=german,
               language=french]{acmart}
\end{verbatim}

The title, subtitle, keywords and abstract will be typeset in the main
language of the paper.  The commands \verb|\translatedXXX|, \verb|XXX|
begin title, subtitle and keywords, can be used to set these elements
in the other languages.  The environment \verb|translatedabstract| is
used to set the translation of the abstract.  These commands and
environment have a mandatory first argument: the language of the
second argument.  See \verb|sample-sigconf-i13n.tex| file for examples
of their usage.

\section{SIGCHI Extended Abstracts}

The ``\verb|sigchi-a|'' template style (available only in \LaTeX\ and
not in Word) produces a landscape-orientation formatted article, with
a wide left margin. Three environments are available for use with the
``\verb|sigchi-a|'' template style, and produce formatted output in
the margin:
\begin{description}
\item[\texttt{sidebar}:]  Place formatted text in the margin.
\item[\texttt{marginfigure}:] Place a figure in the margin.
\item[\texttt{margintable}:] Place a table in the margin.
\end{description}

%%
%% The acknowledgments section is defined using the "acks" environment
%% (and NOT an unnumbered section). This ensures the proper
%% identification of the section in the article metadata, and the
%% consistent spelling of the heading.
\begin{acks}
To Robert, for the bagels and explaining CMYK and color spaces.
\end{acks}

%%
%% The next two lines define the bibliography style to be used, and
%% the bibliography file.
\bibliographystyle{ACM-Reference-Format}
\bibliography{report}


%%
%% If your work has an appendix, this is the place to put it.
\appendix

\section{Research Methods}

\subsection{Part One}

Lorem ipsum dolor sit amet, consectetur adipiscing elit. Morbi
malesuada, quam in pulvinar varius, metus nunc fermentum urna, id
sollicitudin purus odio sit amet enim. Aliquam ullamcorper eu ipsum
vel mollis. Curabitur quis dictum nisl. Phasellus vel semper risus, et
lacinia dolor. Integer ultricies commodo sem nec semper.

\subsection{Part Two}

Etiam commodo feugiat nisl pulvinar pellentesque. Etiam auctor sodales
ligula, non varius nibh pulvinar semper. Suspendisse nec lectus non
ipsum convallis congue hendrerit vitae sapien. Donec at laoreet
eros. Vivamus non purus placerat, scelerisque diam eu, cursus
ante. Etiam aliquam tortor auctor efficitur mattis.

\section{Online Resources}

Nam id fermentum dui. Suspendisse sagittis tortor a nulla mollis, in
pulvinar ex pretium. Sed interdum orci quis metus euismod, et sagittis
enim maximus. Vestibulum gravida massa ut felis suscipit
congue. Quisque mattis elit a risus ultrices commodo venenatis eget
dui. Etiam sagittis eleifend elementum.

Nam interdum magna at lectus dignissim, ac dignissim lorem
rhoncus. Maecenas eu arcu ac neque placerat aliquam. Nunc pulvinar
massa et mattis lacinia.


\end{document}
\endinput
%%
%% End of file `sample-manuscript.tex'.
